\section{Viterbi algorithm}
We can use dynamic programming on a directed graph $G(V;E)$ for speech recognition. Each edge $(u,v)\in E$ is labeled with a sound from a finite set $\Sigma$ of sounds. The labeled graph is a formal model of a person speaking a restricted language. Each path in the graph starting from a distinguished vertex $v_0\in V$ corresponds to a possible sequence of sounds produced by the model. We define the label of a directed path to be the concatenation of the labels of the edges on that path.

\begin{enumerate}
\item Describe a polynomial time algorithm that, given an edge-labeled graph $G$ with distinguished vertex $v_0$ and a sequence $s=\langle\sigma_1,\dots,\sigma_k\rangle$ of sounds from $\Sigma$, returns a path in $G$ that begins at $v_0$ and has $s$ as its label $s$, if such path exists. Otherwise, the algorithm should return NO-SUCH-PATH.

\ \\{\bf Solution:} We can just use a BFS-like algorithm. We start with an queue with a single element $v_0$. Then for each label in the sequence, we pop all the vertices. For each of the popped vertex, we examine the edges connected. If any edge is labeled with $\sigma_i$, we push the end vertex into the queue. During the process, we mark down the ancestor of each vertex. If any time the queue is empty, we return NO-SUCH-PATH. At last, if there are still any vertex in the queue, we can reconstruct the origin path based on the ancestor marked by the vertices.

\item Now, suppose that every edge $(u,v)$ has an associated nonnegative probability $p(u,v)$ of traversing the edge, and thus producing the corresponding sound. The sum of the probabilities of the edges leaving any vertex equals 1. The probability of a path is defined to be the product of the probabilities of its edges. We can view the probability of a path beginning at $v_0$ as the probability that a random walk beginning at $v_0$ will follow the specific path, where we randomly choose which edge to take leaving a vertex $u$ according to the probabilities of the available edges leaving $u$. Extend your answer to part (a) so that if a path is returned, it is a most probable path starting at $v_0$ and having label $s$. Analyze the running time of your algorithm.

\ \\{\bf Solution:} The difference is that we store the maximum likelihood and the corresponding ancestor when we expand a node to its descendent. The running time is the same as before (?).
\end{enumerate}